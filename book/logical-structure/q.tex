% !TeX root = ../../infdesc.tex
\begin{chapex}
\label{cqTranslatePropositionalFormulaeToEnglish}
For fixed $n \in \mathbb{N}$, let $p$ represent the proposition `$n$ is even', let $q$ represent the proposition `$n$ is prime' and let $r$ represent the proposition `$n = 2$'. For each of the following propositional formulae, translate it into plain English and determine whether it is true for all $n \in \mathbb{N}$, true for some values of $n$ and false for some values of $n$, or false for all $n \in \mathbb{N}$.

\begin{enumerate}[(a)]
\item $(p \wedge q) \Rightarrow r$
\item $q \wedge (\neg r) \Rightarrow (\neg p)$
\item $((\neg p) \vee (\neg q)) \vee (\neg r)$
\item $(p \wedge q) \wedge (\neg r)$
\end{enumerate}
\end{chapex}

\begin{chapex}
For each of the following plain English statements, translate it into a symbolic propositional formula. The propositional variables in your formulae should represent the simplest propositions that they can.
\begin{enumerate}[(a)]
\item Guinea pigs are quiet, but they're loud when they're hungry.
\item It doesn't matter that $2$ is even, it's still a prime number.
\item $\sqrt{2}$ can't be an integer because it is an irrational number.
\end{enumerate}
\end{chapex}

\begin{chapex}
Let $p$ and $q$ be propositions, and assume that $p \Rightarrow (\neg q)$ is true and that $(\neg q) \Rightarrow p$ is false. Which of the following are true, and which are false?
\begin{enumerate}[(a)]
\item $q$ being false is necessary for $p$ to be true.
\item $q$ being false is sufficient for $p$ to be true.
\item $p$ being true is necessary for $q$ to be false.
\item $p$ being true is sufficient for $p$ to be false.
\end{enumerate}
\end{chapex}

In \Crefrange{cqStructureProofBegin}{cqStructureProofEnd}, use the definitions of the logical operators in \Cref{secPropositionalLogic} to describe what steps should be followed in order to prove the propositional formula in the question; the letters $p$, $q$, $r$ and $s$ are propositional variables.

\begin{chapex}
\label{cqStructureProofBegin}
$(p \wedge q) \Rightarrow (\neg r)$
\end{chapex}

\begin{chapex}
$(p \vee q) \Rightarrow (r \Rightarrow s)$
\end{chapex}

\begin{chapex}
$(p \Rightarrow q) \Leftrightarrow (\neg p \Rightarrow \neg q)$
\end{chapex}

\begin{chapex}
\label{cqStructureProofEnd}
$(p \wedge (\neg q)) \vee (q \wedge (\neg p))$
\end{chapex}

% Variables and quantifiers %

\begin{chapex}
Find a statement in plain English, involving no variables at all, that is equivalent to the logical formula $\forall a \in \mathbb{Q},\, \forall b \in \mathbb{Q},\, (a < b \Rightarrow \exists c \in \mathbb{R},\, [a < c < b ~\wedge~ \neg (c \in \mathbb{Q})])$. Then prove this statement, using the structure of the logical formula as a guide.
\end{chapex}

\begin{chapex}
Find a purely symbolic logical formula that is equivalent to the following statement, and then prove it: ``\textit{No matter which integer you may choose, there will be an integer greater than it.}''
\end{chapex}

\begin{chapex}
Let $X$ be a set and let $p(x)$ be a predicate. Find a logical formula representing the statement `there are exactly two elements $x \in X$ such that $p(x)$ is true'. Use the structure of this logical formula to describe how a proof should be structured, and use this structure to prove that there are exactly two real numbers $x$ such that $x^2=1$.
\end{chapex}

% Logical equivalence %

\begin{chapex}
Prove that
\[
p \Leftrightarrow q \equiv (p \Rightarrow q) \wedge ((\neg p) \Rightarrow (\neg q))
\]
How might this logical equivalence help you to prove statements of the form `$p$ if and only if $q$'?
\end{chapex}

\begin{chapex}
Prove using truth tables that $p \Rightarrow q \not\equiv q \Rightarrow p$. Give an example of propositions $p$ and $q$ such that $p \Rightarrow q$ is true but $q \Rightarrow p$ is false.
\end{chapex}

In \Crefrange{cqTruthTablesBegin}{cqTruthTablesEnd}, find a logical formula whose column in a truth table is as shown.

\begin{chapex}
\label{cqTruthTablesBegin}
\begin{tabular}{cc|c}
$p$ & $q$ & \hspace{50pt} \\ \hline
\TT & \TT & \FF \\
\TT & \FF & \TT \\
\FF & \TT & \TT \\
\FF & \FF & \FF \\
\end{tabular}
\end{chapex}

\begin{chapex}
\begin{tabular}{cc|c}
$p$ & $q$ & \hspace{50pt} \\ \hline
\TT & \TT & \TT \\
\TT & \FF & \FF \\
\FF & \TT & \TT \\
\FF & \FF & \FF \\
\end{tabular}
\end{chapex}

\begin{chapex}
\begin{tabular}{ccc|c}
$p$ & $q$ & $r$ & \hspace{50pt} \\ \hline
\TT & \TT & \TT & \TT \\
\TT & \TT & \FF & \TT \\
\TT & \FF & \TT & \FF \\
\TT & \FF & \FF & \FF \\
\FF & \TT & \TT & \FF \\
\FF & \TT & \FF & \FF \\
\FF & \FF & \TT & \TT \\
\FF & \FF & \FF & \TT \\
\end{tabular}
\end{chapex}

\begin{chapex}
\label{cqTruthTablesEnd}
\begin{tabular}{ccc|c}
$p$ & $q$ & $r$ & \hspace{50pt} \\ \hline
\TT & \TT & \TT & \TT \\
\TT & \TT & \FF & \FF \\
\TT & \FF & \TT & \FF \\
\TT & \FF & \FF & \FF \\
\FF & \TT & \TT & \TT \\
\FF & \TT & \FF & \FF \\
\FF & \FF & \TT & \TT \\
\FF & \FF & \FF & \FF \\
\end{tabular}
\end{chapex}

\begin{chapex}
A new logical operator $\uparrow$ is defined by the following rules:
\begin{enumerate}[(i)]
\item If a contradiction can be derived from the assumption that $p$ is true, then $p \uparrow q$ is true;
\item If a contradiction can be derived from the assumption that $q$ is true, then $p \uparrow q$ is true;
\item If $r$ is any proposition, and if $p \uparrow q$, $p$ and $q$ are all true, then $r$ is true.
\end{enumerate}

This question explores this curious new logical operator.
\begin{enumerate}[(a)]
\item Prove that $p \uparrow p \equiv \neg p$, and deduce that $((p \uparrow p) \uparrow (p \uparrow p)) \equiv p$.
\item Prove that $p \vee q \equiv (p \uparrow p) \uparrow (q \uparrow q)$ and $p \wedge q \equiv (p \uparrow q) \uparrow (p \uparrow q)$.
\item Find a propositional formula using only the logical operator $\uparrow$ that is equivalent to $p \Rightarrow q$.
\end{enumerate}
\end{chapex}

\begin{chapex}
Let $X$ be $\mathbb{Z}$ or $\mathbb{Q}$, and define a logical formula $p$ by:
$$\forall x \in X,\, \exists y \in X,\, (x<y \wedge [\forall z \in X,\, \neg (x<z \wedge z<y)])$$
Write out $\neg p$ as a maximally negated logical formula. Prove that $p$ is true when $X = \mathbb{Z}$, and $p$ is false when $X = \mathbb{Q}$.
\end{chapex}

\begin{chapex}
Use \Cref{defUniqueExistentialQuantifier} to write out a maximally negated logical formula that is equivalent to $\neg \exists! x \in X,\, p(x)$. Describe the strategy that this equivalence suggests for proving that there is not a unique $x \in X$ such that $p(x)$ is true, and use this strategy to prove that, for all $a \in \mathbb{R}$, if $a \ne -1$ then there is not a unique $x \in \mathbb{R}$ such that $x^4-2ax^2+a^2-1=0$.
\end{chapex}

\begin{chapex}
Define a new quantifier $\forall !$ such that de Morgan's laws for quantifiers (\Cref{thmDeMorganQuantifiers}) hold with $\forall$ and $\exists$ replaced by $\forall !$ and $\exists !$, respectively.
\end{chapex}

\subsection*{True--False questions}

\tfquestiontext{cqLogicTFBegin}{cqLogicTFEnd}

\begin{chapex} % True
\label{cqLogicTFBegin}
Every implication is logically equivalent to its contrapositive.
\end{chapex}

\begin{chapex} % False
Every implication is logically equivalent to its converse.
\end{chapex}

\begin{chapex} % True
Every propositional formula whose only logical operators are conjunctions and negations is logically equivalent to a propositional formula whose only logical operators are disjunctions and negations.
\end{chapex}

\begin{chapex} % False
Every propositional formula whose only logical operators are conjunctions is logically equivalent to a propositional formula whose only logical operators are disjunctions.
\end{chapex}

\begin{chapex} % False
The formulae $p \wedge (q \vee r)$ and $(p \wedge q) \vee r$ are logically equivalent.
\end{chapex}

\begin{chapex} % True
The formulae $p \vee (q \vee r)$ and $(p \vee q) \wedge (p \vee r)$ are logically equivalent.
\end{chapex}

\begin{chapex} % False
The logical formulae $\neg \forall x \in X,\, \forall y \in Y,\, p(x,y)$ and $\exists x \in X,\, \forall y \in Y,\, p(x,y)$ are logically equivalent.
\end{chapex}

\begin{chapex} % False
$\neg \forall x \ge 0,\, \exists y \in \mathbb{R},\, y^2=x$ is logically equivalent to $\forall x < 0,\, \exists y \not\in \mathbb{R},\, y^2 \ne x$.
\end{chapex}

\begin{chapex} % True
$\neg \forall x \ge 0,\, \exists y \in \mathbb{R},\, y^2=x$ is logically equivalent to $\exists x \ge 0,\, \forall y \in \mathbb{R},\, y^2 \ne x$.
\end{chapex}

\begin{chapex} % False
\label{cqLogicTFEnd}
$\neg \forall x \ge 0,\, \exists y \in \mathbb{R},\, y^2=x$ is logically equivalent to $\exists x < 0,\, \forall y \in \mathbb{R},\, y^2 = x$.
\end{chapex}

\subsection*{Always--Sometimes--Never questions}

\asnquestiontext{cqLogicASNBegin}{cqLogicASNEnd}

\begin{chapex} % Sometimes
\label{cqLogicASNBegin}
Let $p$ and $q$ be propositions and assume that $p$ is true. Then $p \Rightarrow q$ is true.
\end{chapex}

\begin{chapex} % Always
Let $p$ and $q$ be propositions and assume that $p$ is false. Then $p \Rightarrow q$ is true.
\end{chapex}

\begin{chapex} % Always
Let $X$ and $Y$ be sets and let $p(x,y)$ be a predicate with free variables $x \in X$ and $y \in Y$. Then the logical formulae $\forall x \in X,\, \forall y \in Y,\, p(x,y)$ and $\forall y \in Y,\, \forall x \in X,\, p(x,y)$ are logically equivalent.
\end{chapex}

\begin{chapex} % Sometimes
Let $X$ and $Y$ be sets and let $p(x,y)$ be a predicate with free variables $x \in X$ and $y \in Y$. Then the logical formulae $\forall x \in X,\, \exists y \in Y,\, p(x,y)$ and $\exists y \in Y,\, \forall x \in X,\, p(x,y)$ are logically equivalent.
\end{chapex}

\begin{chapex} % Never
\label{cqLogicASNEnd}
Let $X$ and $Y$ be sets and let $p(x,y)$ be a predicate with free variables $x \in X$ and $y \in Y$. Then the logical formulae $\forall x \in X,\, \forall y \in Y,\, p(x,y)$ and $\exists x \in X,\, \exists y \in Y,\, \neg p(x,y)$ are logically equivalent.
\end{chapex}