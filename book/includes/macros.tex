% !TeX root = ../../infdesc.tex

%
% Arithmetic operations
%

% Floor and ceiling
\newcommand{\floor}[1]{\lfloor {#1} \rfloor}
\newcommand{\dfloor}[1]{{\displaystyle\left\lfloor {#1} \right\rfloor}}
\newcommand{\ceil}[1]{\lceil {#1} \rceil}
\newcommand{\dceil}[1]{{\displaystyle\left\lceil {#1} \right\rceil}}

%
% Sets and set operations
%

% Number sets

% Font for number sets
% - Note: the blackboard bold font is referred to explicitly in the
%   textbook, so it is best not to change this.
\newcommand{\numsetfont}{\mathbb}

\newcommand{\Nat}{\numsetfont{N}} % Natural numbers
\newcommand{\Int}{\numsetfont{Z}} % Integers
\newcommand{\Rat}{\numsetfont{Q}} % Rational numbers
\newcommand{\Real}{\numsetfont{R}} % Real numbers
\newcommand{\Cplx}{\numsetfont{C}} % Complex numbers

% Set-builder notation
\newcommand{\setb}[2]{\{ #1 ~|~ #2 \}}
\newcommand{\dsetb}[2]{\left\{ #1 ~\middle|~ #2 \right\}}
\newcommand{\setbsizemod}[3]{#3\{ #1 ~#3|~ #2\}}
\newcommand{\bigsetb}[2]{\setbsizemod{#1}{#2}{\big}}
\newcommand{\Bigsetb}[2]{\setbsizemod{#1}{#2}{\Big}}
\newcommand{\biggsetb}[2]{\setbsizemod{#1}{#2}{\bigg}}
\newcommand{\Biggsetb}[2]{\setbsizemod{#1}{#2}{\Bigg}}

% Set operations
\newcommand{\power}{\mathcal{P}} % Power set
\newcommand{\relcomp}{\setminus} % Relative complement